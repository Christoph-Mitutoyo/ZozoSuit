\chapter{Stand der Technik}
\label{ch:sdt}

\section{Frontend}

In dieser Studienarbeit sollen Bilder von einer Person im ZOZOSUIT gemacht werden und diese anschließend ausgewertet und als 3D-Modell dargestellt werden. Da sowohl Bilder 
gemacht, als auch Grafiken angezeigt werden, bietet sich das Smartphone als Frontend an. 

In Deutschland besitzen 81\% der Bevökerung ab 14 Jahren ein Smartphone, in jungen Altersgruppen steigt der Prozentsatz auf über 95\% \cite{misc:marktforschung_smartphone}. 
Für Smartphones bestehen zwei dominante Betriebssysteme: Android und i\acrshort{os}. Mit einem Marktanteil von 69,8\% ist Android in Deutschland Marktführer für 
Smartphonebetriebssysteme, gefolgt von i\acrshort{os} mit 29,8\%. Andere Betriebssysteme, wie Windows und Blackberry, 
besitzen einen Marktanteil von 0,4\%. \cite{misc:kantarworldpanel}. \newline
Für Android entwickelte Apps basieren auf Java oder Kotlin, während i\acrshort{os}-Apps in Swift oder Objective-C entwickelt werden. Des Weiteren gibt es Tools und \glspl{sdk}, 
mit welchen Apps entwickelt werden können, welche mit wenigen Einschränkungen auf beiden Betriebssystemen lauffähig sind:
\begin{itemize}
    \item \textbf{React Native}: React Native ist ein JavaScript Framework, welches die \gls{ui} in native (Android oder iOS spezifische) Elemente umwandelt. Die Logik bleibt dabei unverändert. Das Framework wird von Facebook, Instagram und Uber benutzt.
    \item \textbf{Xamarin}: Xamarin ermöglicht es Entwicklern eine gemeinsame Logik für Android und iOS zu schreiben. Die jeweilige UI wird allerdings in einer nativen Programmiersprache entwickelt.
    \item \textbf{Flutter}: Flutter ist ein \gls{sdk}, welches von Google erstellt, und im Jahr 2018 erstmals in der Version 1.0 veröffentlicht wurde. Das \gls{sdk} verwendet die, ebenso von Google entwickelte, Programmiersprache Dart. Flutter ermöglicht es UI-Komponenten zu entwickeln, welche auf beiden Betriebssystemen konsistent sind.
\end{itemize}

Damit Apps im App Store von Apple veröffentlicht werden dürfen, benötigt der/die Entwickler/-in eine Mitgliedschaft im Apple Developer Program. Diese kostet pro Jahr 99 US-Dollar \cite{misc:appledeveloper}.
Des Weiteren kann eine für i\gls{os} entwickelte Anwendung nur in einer Apple-Umgebung (Macbook, \dots) kompiliert werden. \newline
Aus diesen Gründen, und wegen dem Verbreitungsanteil von 29,8\% in Deutschland, ist das Entwickeln dieser App für eine reine i\gls{os}-Umgebung nicht sinnvoll. \newline

Um Apps im Google Play Store zu veröffentlichen, wird eine einmalige Registrierungsgebühr von 25 US-Dollar benötigt \cite{misc:androiddeveloper}. Eine jährliche Gebühr ist 
nicht vorhanden.

React Native ermöglicht zwar das Entwickeln einer Anwendung für iOS und Android, jedoch werden nicht alle plattformspezifischen \glspl{api} unterstützt. Nicht unterstützte 
\glspl{api} müssen in der nativen Programmiersprache erstellt werden. Es muss somit trotzdem für iOS und Android separat entwickelt werden. \cite{misc:reactnative_vs_native}\newline
Die nicht vollständige Unterstützung nativer \glspl{api} ist eine Schwachpunkt aller Frameworks, welche das gleichzeitige Entwickeln für iOS und Android ermöglichen. \newline
Xamarin bietet durch die Unterstützung von .Net und Microsoft Visual Studio die beste Entwicklungsumgebung. Jedoch mindert die geringe Popularität die verfügbaren Hilfestellungen 
durch andere Entwickler/-innen \cite{misc:flutter_reactnative_xamarin}. \newline
Laut einer Umfrage von Stack Overflow ist Flutter das beliebteste der drei vorgestellten Frameworks \cite{misc:so_popularity}. Es ist jedoch auch das jüngste 
(Version 1.0 im Jahr 2018) und kann somit auf den geringsten Anteil an verfügbaren Hilfestellungen zurückgreifen. Auch die Unterstützung durch Bibliotheken von Dritten ist 
noch nicht so ausgereift, wie bei den anderen Frameworks. Allerdings steht eine gute Online-Dokumentation zur Verfügung, welche unter \cite{misc:flutter_docs} erreichbar ist.
\cite{misc:flutter_reactnative_xamarin}

Ein weiterer negativer Aspekt der Cross-Platform-Entwicklung sind Systemupdates von iOS oder Android. Ein Systemupdate kann neue Funktionen hinzufügen und alte Verändern 
oder Entfernen. Während bei nativen Plattformen darauf geachtet wird, dass alle Änderungen rückwärtskompatibel sind, müssen sich Cross-Platform-Frameworks erst an die 
Änderungen anpassen. Dadurch kann es während Übergangsphasen zu fehlerhaftem Verhalten der App kommen.

In Anbetracht dieser Punkte ist das Entwickeln des Frontends als reine Android-App oder als plattformübergreifende App unter Verwendung eines Frameworks denkbar. Die Wahl fällt hierbei
auf eine Cross-Platform-App, welche mithilfe von Flutter, und der damit verbundenen Programmiersprache Dart, erstellt wird. Dies begründet sich darin, dass bei einer reinen Android-
Anwendung knapp ein Drittel der Smartphonebesitzer keinen Zugriff auf die App haben würden. Die Wahl von Flutter ergibt sich aus der guten Online-Dokumentation und der Popularität, 
durch welche möglicherweise auftretende Probleme besser gelöst werden können. \newline
Als Entwicklungsumgebung wird Android Studio gewählt, da dieses ein Flutter-Plugin besitzt und einen 
Emulator bereitstellt, mit welchem verschieden Smartphones auf verschiedenen \acrshort{api}-Stufen simuliert werden können.
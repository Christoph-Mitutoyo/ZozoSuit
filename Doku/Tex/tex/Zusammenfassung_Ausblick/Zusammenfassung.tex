\chapter{Zusammenfassung und Ausblick}

In dieser Studienarbeit konnte das Ziel der Generierung eines 3D-Modells aus einem Bild einer Person, welche den ZOZOSUIT trägt, realisiert werden. \newline
Es werden allerdings nicht alle Punkte zuverlässig erkannt. Zudem werden zur Generierung des 3D-Modells eines Menschen bestimmte Punkte am Körper benötigt. Der ZOZOSUIT hat nur 500 
verfügbare Punkte, welche bestenfalls in der Nähe der benötigten Punkte liegen. Zudem kann ein nicht perfekt sitzender ZOZOSUIT für starke Verfälschungen sorgen.\newline
Das Ergebnis ist ein 3D-Modell, welches besonders in der Haltung nicht dem ursprünglichen Bild entspricht. Besonders anfällig sind hierbei die Positionen der Arme und Beine.

Der von ZOZO inc. veröffentlichte ZOZOSUIT 2 könnte diese Probleme lösen. So sind bei diesem die zur Identifizierung benötigten Punkte besser erkennbar und die Anzahl der auf dem Suit 
verfügbaren Punkte wurde auf 20000 erhöht. \newline
Durch die erhöhte Punktanzahl und einer besseren Erkennwahrscheinlichkeit, können die für ein 3D-Modell relevanten Punkte zuverlässiger erkannt werden. Zudem soll der ZOZOSUIT 2 passgenauer 
sein, was einer Verschiebung des Anzugs entgegenwirkt.

Eine Wiederaufnahme dieser Studienarbeit mit einem ZOZOSUIT 2 wäre somit sehr interessant.